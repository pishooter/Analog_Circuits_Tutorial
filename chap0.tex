\chapter*{绪论}
\addcontentsline{toc}{chapter}{绪论}
本章旨在为后续各章所讲述的内容进行逻辑上的串联。

\section{模拟电路是什么?}
信号在生活中无处不在,比如温度、声音、光等等。而由于电信号便于传输、处理和控制,于是人们就将各种信号转化为电信号,并发展了一套处理电信号的方法,逐渐形成了一门新的学科——\textbf{电子学}。

在电子电路中,信号分为\textbf{模拟信号}\index{M!模拟信号}和\textbf{数字信号}\index{S!数字信号}。模拟信号指的是在时间和数值上均连续的信号,而数字信号指的是在时间和数值上均离散的信号。所谓模拟电路,就是处理模拟信号的电路。

\section{将要学习哪些内容?}
总的来说,模拟电路将自下而上地讲解如何处理微小信号。具体分为以下内容:

1.基本的\textbf{半导体材料};

2.由半导体材料构成的\textbf{半导体器件}及其基本特性、基本工作方式;

3.由半导体器件构成的\textbf{基本放大电路};

4.由基本放大电路构成的\textbf{多级放大电路};

5.多级放大电路的\textbf{反馈特性};

6.多级放大电路中的\textbf{集成运算放大器},及其中部分基本的单元电路(差分放大电路、电流源电路等);

7.集成运算放大器的应用(加法减法电路、积分微分电路等)。

本手册将基本按照上述顺序展开。