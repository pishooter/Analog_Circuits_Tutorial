\chapter*{写在前面}
\addcontentsline{toc}{chapter}{写在前面}
2022年,科大将原先的电子技术基础(1)和电子技术基础(2)两门总共四学分的课程,合并为了电子技术基础一门三学分的课程。虽然删掉了部分内容,但是整体而言教学节奏较快,内容较多,因此20级的陈翔学长在2022年秋季学期担任曹平老师的课程助教时编写了一本《电基生存指北cx版》\footnote{《电基生存指北cx版》\href{https://github.com/Anony-Minor/Dianji_zhibei-guidance}{GitHub项目链接}}供同学们使用。虽然该手册在细节上还不够完善,但还是受到了同学们的一致好评。

2023年秋季学期,我也担任了该学期曹平老师的课程助教,于是决定着手完善《电基生存指北cx版》,并最终形成了本手册。在此感谢陈翔学长的大力支持,将其所编写的手册开源,这极大地帮助了本手册的编写工作。

本手册并非《电基生存指北cx版》的简单扩充。在原来内容的基础上,本手册还试图将知识体系进行逻辑上的串联,因此加入了个人的一些拙见。本手册还参考了华成英的《模拟电子技术基础(第五版)》和康华光的《电子技术基础-模拟部分(第7版)》,并引用了其中大量的表格和插图。此外,本手册的部分内容来源于平时批改作业中的整理,以及答疑过程中对同学们遇到的问题的总结,感谢2023秋电子技术基础曹平老师班上的同学对本手册的支持。

编写本手册的目的是为了辅助同学们复习,因此在内容上主要是对于考试相关知识点的罗列,难以替代教材对知识体系的完整架构以及老师对于知识细节的讲解。此外,本手册难免存在一些细节上的漏洞,恳请阅读本手册的同学给予批评指正。

最后,祝同学们学有所获!

\begin{flushright}
    \textsl{2023秋电子技术基础课程助教\quad 罗钰涵\footnote{E-mail:\href{mailto:luoyuhan@mail.ustc.edu.cn}{luoyuhan@mail.ustc.edu.cn}}\\2023年12月}
\end{flushright}




